\documentclass[fleqn]{amsart}
\usepackage{lmodern}
\usepackage[T1]{fontenc}
%\usepackage[utf8]{inputenc}
\usepackage{amssymb}
\usepackage{amsthm}
\usepackage{thmtools}
\usepackage{enumitem}
\usepackage{todonotes}
\usepackage{hyperref}
\usepackage[backend=biber,style=alphabetic,isbn=false,doi=false,url=false]{biblatex}
\renewbibmacro{in:}{}
% \renewcommand{\bibfont}{\raggedright}
%\addbibresource{papersbiblatex.bib} 

\setlist[enumerate,1]{label=\itshape\alph*\upshape)}
\setlist[enumerate,2]{label=\itshape\roman*\upshape)}

\theoremstyle{plain}
\newtheorem{theorem}{Theorem}
\newtheorem{lemma}[theorem]{Lemma}
\newtheorem{proposition}[theorem]{Proposition}
\newtheorem{corollary}[theorem]{Corollary}

\theoremstyle{definition}
\newtheorem{definition}[theorem]{Definition}

\theoremstyle{remark}
\newtheorem{remark}[theorem]{Remark}
\newtheorem{example}[theorem]{Example}

\DeclareMathOperator{\Gal}{Gal}
\DeclareMathOperator{\GL}{GL}
\DeclareMathOperator{\End}{End}
\DeclareMathOperator{\ord}{ord}
%\DeclareMathOperator{\Norm}{Norm}
\DeclareMathOperator{\id}{id}
\DeclareMathOperator{\li}{li}
\DeclareMathOperator{\Div}{Div}
\newcommand{\sep}{\mathrm{sep}}
\newcommand{\ab}{\mathrm{ab}}
\newcommand{\tors}{\mathrm{tors}}
\newcommand{\Norm}[1]{\left\lVert #1 \right\rVert}

\renewcommand{\vec}[1]{\mathbf{#1}}


\DeclareMathOperator{\Oh}{O}
\DeclareMathOperator{\oh}{o}

\title{GPTQ notes}
\author{David Tweedle}
\date{\today}

\begin{document}
\pagestyle{plain}
\begin{abstract}

\end{abstract}
\maketitle
\section{Introduction}
\label{sec:introduction}

Let \(\vec{v}\in\mathbb{R}^n\).
Let \(\vec{x}_1,\ldots, \vec{x}_m\in\mathbb{R}^n\).
Let \(X = 
\begin{pmatrix}
\vec{x}_1 & \vec{x}_2 & \cdots & \vec{x}_m
\end{pmatrix}\).

We are given a quantization function \(Q\).
We try to find \(\tilde{v_2}, \ldots, \tilde{v_n}\) such that with
\[ \tilde{\vec{v}} = (Q(v_1), \tilde{v_2},\ldots, \tilde{v_n}) \]
we have
\[ \sum_i \left|(\vec{v} - \tilde{\vec{v}})^T \vec{x}_i\right|^2 \]
is minimized.

A summand in the above is of the form
\[ (v_1 - Q(v_1))x_1 + (v_2 - \tilde{v_2}) x_2 + \cdots + (v_n - \tilde{v_n})x_n. \]

Setting \(\delta_i = v_i - \tilde{v_i}\) for \(i\geq 2\), we want to find the least squares solution of
\[ (X_2, \ldots, X_n) \delta = (v_1 - Q(v_1))\cdot X_1 \]

One way of proceeding is to multiply on the left by \((X_2, \ldots, X_n)^T\) to get
\[ H_{ij} = X_i^T X_j \] and solve
\[ H\delta = (X_1^T X_2, \ldots, X_1^T X_n)^T (v_1 - Q(v_1)) \]

Letting \(H\) represent the full matrix including the first column of \(X\), GPTQ proceeds by computing the inverse of \(H + \lambda\) using the Cholesky factorization.

We propose writing
\[ X = USV^T \]
Then
\[ X_1 = USV_1^T, \] and 
\[ \bar{X} = US\bar{V}^T \]
and now we try to solve
\[ US\bar{V}^T \delta = USV_1^T(v_1 - Q(v_1)) \]
% here U is m by d, S is d by d, V is n by d
% 
so \(\delta\) is found:
\[ \delta = \bar{V}V_1^T (v_1 - Q(v_1)). \]

\end{document}
